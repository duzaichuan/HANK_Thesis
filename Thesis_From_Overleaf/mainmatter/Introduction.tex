




\section{Introduction}
\label{chap:intro}






%\subsection{Existing Literature}
%\subsubsection{Automatic Stabilizers}
%\citet{mckay2016role}
%\citet{mckay2016optimal}




%\subsubsection{HANK Models}
\subsection{Heterogeneity in Modern Macroeconomics: HANK Models}
Household heterogeneity has a long tradition in macroeconomics, but has only recently moved seriously into the analyses of business cycles.\footnote{I do not count two-agent models featuring one Ricardian agent and on rule-of-thumb agent as in \citet{campbell1989consumption} as heterogeneous agent models.}\footnote{Inspired by Ben Moll's interview with David Beckworth found \hyperlink{https://www.mercatus.org/bridge/commentary/ben-moll-basics-hank-models-and-how-they-can-be-applied-policymaking}{here}. See also \hyperlink{https://beatricecherrier.wordpress.com/2018/11/28/heterogeneous-agent-macroeconomics-has-a-long-history-and-it-raises-many-questions/}{this rundown} by Beatrice Cherrier.} The emergence of Heterogeneous Agent New Keynesian models is relatively recent.

The first generation of household heterogeneity on which modern HANK models build derives from, in particular, the seminal papers of \citet{bewley1976permanent}, \citet{huggett1993risk}, \citet{aiyagari1994uninsured}, and \citet{krusell1998income}. The defining feature of these heterogeneous agent models is that they assume incomplete markets. For households, this imply that there is not a complete set of Arrow-Debreu state contingent claims to fully insure against risk. If households are subjected to different shocks this implies different consumption/savings paths, even when conditioning on initial conditions. The latter two papers widened the literature by also implementing heterogeneous households in general equilibrium models. The recent advances, which merges these types of household models with the New-Keynesian models often used for business cycle analysis (\text{HANK}) attempt to formulate models that are consistent with the emergence of rich microeconomic evidence on household's balance sheet and general behavior. 

The introduction of heterogeneous households has a great deal of implications for general equilibrium NK models. These include, but are not by any extent limited to, predictions related to marginal propensities to consume (MPCs), the risk-free rate, Ricardian equivalence, equilibrium determinacy, precautionary savings, credit constraints. Furthermore, the introduction of heterogeneity allows for the analysis of distributional effects, something which representative agent models per definition cannot do. 




%Firstly, the textbook 3 equation NK model is itself relatively young, primarily owing to \citet{goodfriend1997new}, \citet{clarida1999science}, \citet{woodford2003interest}. This model was conceived by merging the real business cycle models of \citet{kydland1982time} with nominal rigidities and market power of firms from the Keynes paradigm. Following the initial appearance of the NK models several medium/large scale versions were later constructed, see \citet{smets2003estimated}, \citet{smets2007shocks}. These usually extended the canonical NK model by have capital accumulation, investment adjustment costs, variable capital utilization, sticky wages and so forth.

 
HANK models have already been applied to a wide amount of central macroeconomic questions such as the effects of monetary policy (\citet{kaplan2018monetary},  \citet{auclert2020micro})...

%Auclert etc: 
%\citet{auclert2018inequality} % Inequality and aggregate demand
%\citet{auclert2019using} % SHADE 
%\citet{auclert2018intertemporal} % The intertemporal keynesian cross
%\citet{auclert2020micro} % Micro Jumps, Macro Humps: monetary policy and business cycles in an estimated HANK model

%Kaplan: 
%\citet{kaplan2018monetary} % Monetary policy according to HANK

%Ravn: 
%\citet{ravn2016macroeconomic} % Macroeconomic fluctuations with HANK \& SAM: An analytical approach

%\subsection{Unemployment Risk}
% interaction between HANK SAM and finding rates/precautionary savings 
 
% \citet{carroll1997buffer},
 
 



\subsection{Stabilization and The Welfare Cost of Business Cycles.} One of the main objectives of the business cycle literature prevalent in modern macro economics is to characterize ways to stabilize economies subject to these cycles. 
The need for stabilization is usually argued by considering the negative effects of declines in output, consumption and employment and the resulting effects on welfare during recessions. Still, some have argued that the welfare costs arising from cycles are so marginal that policy intuitions are better of focusing on long run, supply side policy. This among others the argument of Nobel laureate Robert Lucas Jr. (\citet{lucas2003macroeconomic}). Other researchers find more significant efficiency losses arsing from business cycles (see \citet{gali2007markups}, \citet{tella2003macroeconomics} etc.). 


In brief, the welfare losses can be decomposed into three channels: A consumption channel, an uncertainty channel, and an effort/well-being related channel. The first channel captures the direct welfare loss which inevitably occurs during a recession when production, income and thus consumption declines. The second channel refers to the fact that consumers are generally risk averse, and thus incur welfare losses when income streams are volatile. Technically, the first channel arises from first-order effects on utility and the second channel arises from second-order effects. The third channel relates to subjective well-being, the most prominent one being that of the negative psychological consequences of unemployment spells (\citet{wolfers2003business}).

Assume that the above channels add up to such large welfare losses that stabilization is needed. The most studied instruments in the literature in dealing with economic fluctuations are monetary and government policy. Both appear in discretionary versions, but also adhere to certain rules. For monetary policy, the Taylor rule is well known. For government policy, various rules of law act as automatic stabilizers to the economy. The most well-known examples include unemployment benefits, progressive income taxation, corporate taxes etc. Unemployment benefits reduce fluctuations by lessening the income change that occur when people move in and out of employment. Similarly, progressive taxation reduce income fluctuations since disposable income is a convex function of taxable income under such a system.  Stabilizing income in turn stabilizes consumer demand thus reducing economic fluctuations. This type of stabilizers generally imply that the government's budget balance is procyclical, other things equal...   
%ADD INTERACTION BETWEEN HANK AND AUTOMATIC STAB. HERE

%The main subject of this thesis is the valuation of these automatic stabilizers in reducing business cycle fluctuations. The main novelty is that the frame of analysis is a fully fleshed out heterogeneous agent New Keynesian (HANK) model. Using heterogeneous agent (HA) models as opposed to representative agent (RA) has importance applications for the impact of automatic stabilizers. HA models enhance models with a number of features, but with respect to analyzing automatic stabilizers the most important feature is the presence of endogenous hand-to-mouth (HtM) consumers. Recall that HtM consumers denote consumers that simply consume all their income and thus holds no savings. This kind of behavior tend to occur for households that credit constrained. Since the average household is not credit constrained HtM behavior does not appear in true RA models. A wide class of models contain a "standard" Ricardian household and a HtM household to mimic this credit constrained behavior. Typically, the HtM household is modelled as a consumer that simply "eats" his/hers entire income each period. 

%The importance of this in relation automatic stabilizers is straightforward. During a recession more agents become unemployed. This implies large drops in income, and agents that before the recession had small amounts of savings might become credit constrained now and thus essentially move in HtM territory. Since HtM agents has large marginal propensities to consume...
% Ricardian Equiv. 