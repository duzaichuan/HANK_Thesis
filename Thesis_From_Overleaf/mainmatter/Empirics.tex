




\section{Automatic Stabilizers in the Data}
\label{chap:Emprirics}
It is infeasible to model the entirety of the danish tax/transfer system. Thus, I will select a few, but important stabilizers based on their correlation with the business cycle in the data.

\citet{cohen1999automatic}, 


\subsection{Government Budget}
In core macroeconomic models, the government's budget balance is often simple, thus containing only a few taxes and transfers. In reality it contains many posts, though it is dominated by a few large posts. Table \ref{GBB_components} presents the major posts of the Danish government budget balance along with their average share of GDP for the period 1971-2017.  


\bgroup
\def\arraystretch{1.2}
\setlength{\tabcolsep}{1.2em} 
\begin{table}[H]
\tiny
%\captionsetup{font=footnotesize}
\caption{Components of Danish Government Budget Balance, 1971-2017} 
\label{GBB_components}
\centering
\footnotesize
\begin{threeparttable}
\scriptsize 
\makebox[\textwidth]{ 
\begin{tabular}{lc  lc}
\toprule
\multicolumn{2}{c}{{Revenues}} & \multicolumn{2}{c}{Expenses} \\ 
\cmidrule(lr){1-2} \cmidrule(lr){3-4} 
 {Post}   & Avg. GDP Share (\%)  & Post & Avg. GDP Share (\%)    \\
\cmidrule(lr){1-1} \cmidrule(lr){2-2} \cmidrule(lr){3-3} \cmidrule(lr){4-4} 
Direct Taxes  &  27.5  & Unemployment Insurance & 1.88   \\
- Income Taxes  &  21.2  &  Early Retirement & 1.03   \\
- Labour Market Contribution  &  4.20  &   Leave Benefits & 0.08    \\
- Corporate Taxes  &  2.15  &  Pensions & 12.9     \\
- Pension + Capital Taxes  & 1.36  &  Transfers to private institutions & 0.79    \\
- Vehicle  Weight Duty &  0.83 & Public expenses to foreign & 1.56    \\
Indirect Taxes  &  16.4  & Net public transfers to Private Firms & 0.09   \\
- VAT &  9.04 &  Subsidies & 2.07   \\
- Excise Taxes &  4.54 &  Public Consumption & 24.8   \\
- Real Estate Taxes &  1.19  & Public Investment & 3.23   \\
- Vehicle registration tax &  1.12 &  &   \\
- Tariffs &  0.25 &  &   \\
Social Contributions & 1.89 &  & \\
Public Revenues from foreign &  0.07  & &  \\
Net Interest Revenue &  -2.06  & &  \\
Public Reinvestment  &  2.83  &  &  \\
Other Revenues  &  0.98  & &   \\
\cmidrule(lr){1-2} \cmidrule(lr){3-4} 
Total Revenues  &  48.4  & Total Expenses  &   47.6   \\                            
\midrule[\heavyrulewidth]
\end{tabular}
}
\end{threeparttable}
\begin{tablenotes}[para]\footnotesize 
\item {\it Based on quarterly data from the danish central banks MONA database. }      
\end{tablenotes}
\end{table}
 
% tyd : Arbejdsløshedsunderstøttelse
% tye : Efterløn og overgangsydelse
% tyo : Orlovsydelse
% typ : Pensioner
% typi : Transfereringer til private institutioner 
% tenou : Offentlig udgift til udland
% tkon-tiov : Kapital transfereringer fra offentlig sektor, - Overskud af offentlige virksomheder 
% sisub : Subsidier
% pco : Offentligt forbrug, deflator
% fco : Offentligt forbrug
% pio : Offentlig investering, deflator
% fio : Offentlige investeringer  


% siaf : Indirekte skatter i alt 
% sbid : Sociale bidrag
% sd   : Direkte skatter i alt 
% toi : Andre offentlige indtægter 
% tenoi : Offentlig indtægt fra udland 
% tion : Offentlig renteindtægt, netto 
% iov : Offentlig reinvestering  


The taxes and transfers chosen for the model should be the ones that act as important stabilizers over the cycle. In principle this could be just by picking the taxes/transfers with the largest elasticities w.r.t to output-gap. However, the impact of specific tax/transfers as stabilizers cannot be judged solely on this due to general equilibrium effects. A tax/transfer may correlate poorly with output-gap either because it actually stabilizers output, or because it stimulates the economy and thereby reduced its own character. Consider for instance a negative demand shock initially reduces output and employment. Transfers to unemployed goes up initially, but by reducing the negative income shock to the unemployed demand reduces less compared to a situation without unemployment insurance. This in turn reduces the decline in output and employment, thus reducing aggregate unemployment transfers in the general equilibrium.     
% Basically reverse causality? or not really...
To determine sensitivity to the Business cycle I follow the literature in estimating the elasticises of budget balance posts w.r.t to the cyclical part of GDP. This follows, among others, \citet{cohen1999automatic} and
\citet{mourre2014adjusting}. This methodology is regularly applied by the 
European Commission to calculate the cyclically-adjusted budget balance, which is used for fiscal surveillance in the EU.  
%mourre2019semi

Let $g$ denote a given post in the budget balance. Let $y$ denote nominal GDP, and $\bar{y}$ the GDP trend obtained by applying the Hodrick–Prescott filter. Assume that all variables are in logs such that $y_t - \bar{y}_{t}$ is the output-gap at time $t$. The sensitivity of a given budget balance post to the cycle can be estimated by OLS using the model: 
\begin{gather*}
g_{t}=\alpha_{0}+\alpha_{1}\left(y_{t}-\bar{y}_{t}\right)+\alpha_{2}\ln\bar{y}_{t}+\varepsilon_{t}.   
\end{gather*}



\bgroup
\def\arraystretch{1.2}
\setlength{\tabcolsep}{1.2em} 
\begin{table}[H]
\tiny
%\captionsetup{font=footnotesize}
\caption{Estimated Business Cycle Elasticises of the Danish Government Budget Balance} 
\label{GBB_components}
\centering
\footnotesize
\begin{threeparttable}
\scriptsize 
\makebox[\textwidth]{ 
\begin{tabular}{lc  lc}
\toprule
\multicolumn{2}{c}{{Revenues}} & \multicolumn{2}{c}{Expenses} \\ 
\cmidrule(lr){1-2} \cmidrule(lr){3-4} 
 {Post}   & Elasticity  & Post & Elasticity    \\
\cmidrule(lr){1-1} \cmidrule(lr){2-2} \cmidrule(lr){3-3} \cmidrule(lr){4-4} 
Direct Taxes  &  27.5  & Unemployment Insurance & 1.88   \\
- Income Taxes  &  21.2  &  Early Retirement & 1.03   \\
- Labour Market Contribution  &  4.20  &   Leave Benefits & 0.08    \\
- Corporate Taxes  &  2.15  &  Pensions & 12.9     \\
- Pension + Capital Taxes  & 1.36  &  Transfers to private institutions & 0.79    \\
- Vehicle  Weight Duty &  0.83 & Public expenses to foreign & 1.56    \\
Indirect Taxes  &  16.4  & Net public transfers to Private Firms & 0.09   \\
- VAT &  9.04 &  Subsidies & 2.07   \\
- Excise Taxes &  4.54 &  Public Consumption & 24.8   \\
- Real Estate Taxes &  1.19  & Public Investment & 3.23   \\
- Vehicle registration tax &  1.12 &  &   \\
- Tariffs &  0.25 &  &   \\
Social Contributions & 1.89 &  & \\
Public Revenues from foreign &  0.07  & &  \\
Net Interest Revenue &  -2.06  & &  \\
Public Reinvestment  &  2.83  &  &  \\
Other Revenues  &  0.98  & &   \\
\cmidrule(lr){1-2} \cmidrule(lr){3-4} 
Total Revenues  &  48.4  & Total Expenses  &   47.6   \\                            
\midrule[\heavyrulewidth]
\end{tabular}
}
\end{threeparttable}
\begin{tablenotes}[para]\footnotesize 
\item {\it Based on quarterly data from the danish central banks MONA database. }      
\end{tablenotes}
\end{table}


Main problem is simoultanity. Can it be avoided with SVAR? 