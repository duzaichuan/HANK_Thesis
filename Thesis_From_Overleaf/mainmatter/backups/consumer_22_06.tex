




\section{Household and Bellman Equations}
%\subsection{Households}
Let $i$ denote the individual/household. Flow utility is separable in consumption and search effort, and the individual flow utilities are of the CRRA form. Expected lifetime utility is:
\begin{align*}
U_{i,t}&=\sum_{t}\beta_i^{t}\mathbb{E}_t u\left(c_{i,t},s_{i,t}\right)\\ 
&=\sum_{t}\beta_{i}^{t}\mathbb{E}_{t}\left[\frac{c_{i,t}^{1-\frac{1}{\sigma}}}{1-\frac{1}{\sigma}}-\varphi\frac{s_{i,t}^{1+\frac{1}{\nu}}}{1+\frac{1}{\nu}}n_{i,t}\right]
\end{align*} 
$\sigma$ denotes the intertemporal elasticity of substitution, while $\nu$ is the Frisch elasticity of extensive labor supply. The discount factor $\beta_{i}$ is specific and constant to the individual household, but varies across the population. 
The budget constraints of the consumer are:
\begin{gather*}
\left(1+\tau^{c}\right)c_{i,t}+a_{i,t}=\left(1+r\right)a_{i,t-1}+I_{i,t}+T_{i,t}-\tau\left(I_{i,t}\right) \\
I_{i,t}&=w_{t} e_{i,t}n_{i,t}+b e_{i,t}\left(1-n_{i,t}\right),
\end{gather*}
where $I_{i,t}$ is labor/taxable income. When choosing search effort, the household  takes into account the aggregate relation between job searching and employment: 
\begin{align*}
N_{t}^{e}=\left(1-\delta\right)N_{t-1}^{e}+q_{t}S_{t}.     
\end{align*}
such that a marginal increase in search effort increases expected employment by the job finding rate $q_t$. Not counting employment, the state variables of the household's optimisation problem is $\left(\beta_{i},e_{i},a_{i},\Lambda\right)$, where $\Lambda=\left(w,r,q\right)$ denotes aggregate variables. 
The value function of an employed agent is (dropping $i$'s for notational ease):
\begin{align*}
V_{t}^{n=1}\left(\beta,e_t,a_t,\Lambda_t \right)=&\max_{c_{t},a_{t}}\frac{c_{t}^{1-\frac{1}{\sigma}}}{1-\frac{1}{\sigma}} \\
&+\beta\mathbb{E}_{t}\left[\left(1-\delta\right)V_{t+1}^{n=1}\left(\beta,e_{t+1},a_{t+1},\Lambda_{t+1}\right)\right.\\
&\left.+\delta V_{t+1}^{n=0}\left(\beta,e_{t+1},a_{t+1},\Lambda_{t+1}\right)\right]
\end{align*}
I here make an implicit assumption regarding the timing of employment. I assume that if a worker is fired (which occurs with prob. $\delta$), the worker cannot find a job immediately in the same period, and will remain unemployed for the duration of that period. This implies that the probability of being employed in the next period is $1-\delta$ instead of $1-\delta\left(1-q_{t+1}\right)$. 
The Bellman equation  is composed of an immediate gain from consumption along with an expected continuation value. This is composed of two objects: With probability $\delta$ the agent looses his job The complementary probability is $1-\delta$. 
%To gain intuition for this note that this term can be written: 
%\begin{align*}
%\left(1-\delta\right)V_{i,t+1}^{n=1}\left(I_{i,t},a_{i,t+1}\right)+\delta q_{t+1}\left[V_{i,t+1}^{n=1}\left(I_{i,t},a_{i,t+e}\right)-V_{i,t+1}^{n=0}\left(I_{i,t},a_{i,t+e}\right)\right].
%\end{align*}
The Bellman equation for an unemployed agent is:
\begin{align*}
V_{t}^{n=0}\left(\beta,e_{t+1},a_{t+1},\Lambda_{t+1}\right)=&\max_{c_{t},a_{t},s_{t}}\frac{c_{t}^{1-\frac{1}{\sigma}}}{1-\frac{1}{\sigma}}-\varphi\frac{s_{t}^{1+\frac{1}{\nu}}}{1-\nu}\\
&+\beta\left[q_{t}V_{t+1}^{n=1}\left(\beta,e_{t+1},a_{t+1},\Lambda_{t+1}\right)\right. \\
&\left.+\left(1-q_{t}\right)V_{t+1}^{n=0}\left(\beta,e_{t+1},a_{t+1},\Lambda_{t+1}\right)\right]  
\end{align*}




\section{Optimizing Behavior: Euler Equations}
To solve the dynamic programming problem of the agents EGM is used. For this to be applied, the analytical Euler equations governing intertemporal choices must be derived. I start with the Euler equations of an employed agent. Let $\underline{a}$ be the lower bound of assets (borrowing constraint). The Lagrangian is: 
\begin{align*}
\mathcal{L}^{n=1}=\mathbb{E}_{t}\sum_{t}\beta^{t}\frac{c_{t}^{1-\frac{1}{\sigma}}}{1-\frac{1}{\sigma}}+\lambda_{t}^{1}\left[\left(1+r_{t}\right)a_{t-1}+I_{t}+T_{t}-\tau\left(I_{t}\right)-\left(1+\tau^{c}\right)c_{t}-a_{t}\right]+\lambda_{t}^{2}\left[a_{t}-\underline{a}\right].
\end{align*}
This gives rise to the following FOCs: 
\begin{align*}
\frac{\partial\mathcal{L}^{n=1}}{\partial c_{t}}&=\beta^{t}c_{t}^{-\frac{1}{\sigma}}-\lambda_{t} =0 \\
\frac{\partial\mathcal{L}^{n=1}}{\partial a_{t}}&=-\lambda_{t}+\mathbb{E}_{t}\lambda_{t+1}R_{t+1} +\lambda_{t}^{2}= 0
\end{align*}
where $R_{t+1}=1+r_{t+1}$. If the borrowing constraint is violated ($a_{t}<\underline{a}$) then $\lambda_{t}^{2}\neq0$ and consumption/savings follows from the budget constraint with $a_{t}=\underline{a}$. If the borrowing constraint is not violated then $\lambda_{t}^{2}=0$, and optimal consumption follows 
the Euler equation: 
\begin{gather*}
c_{t}^{-\frac{1}{\sigma}}=\beta\mathbb{E}_{t}R_{t+1}c_{t+1}^{-\frac{1}{\sigma}}\\
\Leftrightarrow c_{t}^{-\frac{1}{\sigma}}=\beta\mathbb{E}_{t}R_{t+1}\left[\left(1-\delta\right)\left(c_{t+1}^{n=1}\right)^{-\frac{1}{\sigma}}+\delta\left(c_{t+1}^{n=0}\right)^{-\frac{1}{\sigma}}\right]   
\end{gather*}


For an unemployed agent the Lagrangian is: 
\begin{align*}
\mathcal{L}^{n=0}=\mathbb{E}_{t}\sum_{t}\beta^{t}\left\{ \frac{c_{t}^{1-\frac{1}{\sigma}}}{1-\frac{1}{\sigma}}-\varphi\frac{s_{t}^{1+\frac{1}{\nu}}}{1-\nu}\right\} &+\lambda_{t}^{1}\left[\left(1+r_{t}\right)a_{t-1}+I_{t}+T_{t}-\tau\left(I_{t}\right)-\left(1+\tau^{c}\right)c_{t}-a_{t}\right]\\
&+\lambda_{t}^{2}\left[a_{t}-\underline{a}\right] \\
&+\mu_{t}\left[N_{t}^{e}-\left(1-\delta\right)N_{t-1}^{e}-q_{t}S_{t}\right]
\end{align*}
The FOCs for consumption and assets are the same as above. The Euler equation is: 
\begin{align*}
c_{t}^{-\frac{1}{\sigma}}=\beta\mathbb{E}_{t}R_{t+1}\left[q_{t}\left(c_{t+1}^{n=1}\right)^{-\frac{1}{\sigma}}+\left(1-q_{t}\right)\left(c_{t+1}^{n=0}\right)^{-\frac{1}{\sigma}}\right].
\end{align*}
The FOCs for search effort and employment are: 
\begin{align*}
\frac{\partial\mathcal{L}^{n=0}}{\partial s_{t}}&=-\beta^{t}\varphi s_{t}^{\frac{1}{\nu}}-\mu_{t}q_{t}d_{i,t}+\mathbb{E}_{t}\left[\lambda_{t+1}^{1}\frac{dI_{t+1}}{ds_{t}}+\mu_{t+1}\left(1-\delta\right)q_{t}d_{i,t}\right] = 0 \\
\frac{\partial\mathcal{L}^{n=0}}{\partial n_{t}}&=\lambda_{t}^{1}\frac{dI_{t}^{n_{t}}}{dn_{t}}+\mu_{t}-\mu_{t+1}\left(1-\delta\right)=0
\end{align*}
where $d_{i,t}$ is the mass of household $i$ (this comes from taken the derivative of $S_{t}=\int s_{i,t}dD_{t}\left(i\right)$). 
Combining the FOCs yields the Euler equation for search effort 
\begin{gather*}
%\varphi s_{i,t}^{-\nu}=c_{i,t}^{-\sigma}\left(w_{t}e_{i}-b\right)q_{t}+\beta\frac{\left(1-q_{t+1}\right)}{q_{t+1}}\varphi s_{i,t+1}^{-\nu}\left(1-\delta\right)
\frac{1}{q_{t}d_{i,t}}\left(1+\tau^{c}\right)\varphi s_{t}^{\frac{1}{\nu}}=c_{t}^{-\frac{1}{\sigma}}\left(w_{t}-b\right)e_{t}+\left(1-\delta\right)\beta\mathbb{E}_{t}\left[c_{t+1}^{-\frac{1}{\sigma}}\left[w_{t+1}-b\right]e_{i,t+1}\right]
\end{gather*}

% CHECK FRISH EXPONENT 





\section{Numerical Implementation}
To implement the households problem described in the prior section I use the endogenous grid method (EGM) of \citet{carroll2006method}. This method is preferable to, for instance, the standard method of value function iteration since it avoids root-finding operations by exploiting the analytical first-order conditions.  \\
The problem is solved as follows. Define grids for the state variables $\beta,e_{t},a_{t}$ as:
\begin{gather*}
\beta\in\left\{ \beta^{1},\beta^{2},...,\beta^{\#_{\beta}}\right\}  \\
e_{t}\in\left\{ e^{1},e^{2},...,e^{\#_{e}}\right\}  \\
a_{t}\in\left\{ a^{1},a^{2},...,a^{\#_{a}}\right\} 
\end{gather*}
The grid for $\beta$ is calibrated by fitting the wealth distribution to data. The grid for $e_t$ is set to reflect the variation in wages following \citet{floden2001idiosyncratic} using estimates from Sweden. The grid for $a_t$ is chosen to to reflect the interval in which households can realistically accumulate assets given the calibration of the economy. The lower bound of this grid follows from the borrowing constraint, while the upper bound is somewhat arbitrarily chosen. The number of points in each grid are $\#_{\beta}=5,\#_{e}=11,\#_{a}=200$ so the state space consist of $5\cdot11\cdot200=11.000$ points.   \\
The solution procedure works as follows for each employment state $\left\{ N,U\right\}$. Given an initial guess for the marginal value function $V_{a,t+1}\left(\beta,e_{t},a_{t},\Lambda_{t}\right)\equiv\frac{\partial}{\partial a_{t}}V_{t+1}\left(\beta,e_{t},a_{t},\Lambda_{t}\right)$:
\begin{enumerate}
    \item Combining the Euler equation and the envelope theorem implies $c_{t}^{-\frac{1}{\sigma}}\left(\beta,e_{t},a_{t},\Lambda_{t}\right)=\beta\mathbb{E}_{t}RV_{a,t+1}\left(\beta,e_{t},a_{t},\Lambda_{t}\right)$.\\ Compute this as: $c_{t}^{-\frac{1}{\sigma}}\left(\beta,e_{t},a_{t},\Lambda_{t}\right)=\beta R\sum_{e^{'}}^{\#_{e}}D\left(e_{t}^{'}\left|e_{t}\right.\right)V_{a,t+1}\left(\beta,e_{t},a_{t},\Lambda_{t}\right)$
    
    \item Invert the marginal utility of consumption to obtain optimal consumption at the grid points: $\tilde{c}_{t}\left(\beta,e_{t},a_{t},\Lambda_{t}\right)=\left[c_{t}^{-\frac{1}{\sigma}}\left(\beta,e_{t},a_{t},\Lambda_{t}\right)\right]^{-\sigma}$.
    \item Calculate the endogenous grid by defining cash-on-hand as $m\left(\beta,e_{t},a_{t},\Lambda_{t}\right)=\tilde{c}_{t}\left(\beta,e_{t},a_{t},\Lambda_{t}\right)+a_{t}$. To determine optimal assets $a^{*}$ on grid points interpolate the grid $a_t$ on $m\left(\beta,e_{t},a_{t},\Lambda_{t}\right)$ and evaluate at $m_{t}=I_{t}\left(e_{t}\right)+\left(1+r_{t}\right)a_{t}$.  
    \item Calculate consumption as $c^{*}\left(\beta,e_{t},a_{t},\Lambda_{t}\right)=m_{t}-a^{*}\left(\beta,e_{t},a_{t},\Lambda_{t}\right)$.
    \item Check whether borrowing constraints bind. If $a^{*}<\underline{a}$ set $a^{*}=\underline{a}$ and recalculate consumption as $c^{*}\left(\beta,e_{t},a_{t},\Lambda_{t}\right)=m_{t}$. 
    \item Calculate the marginal value of assets for further iteration: $V_{a-1,t}\left(\beta,e_{t},a_{t},\Lambda_{t}\right)=c^{*}\left(\beta,e_{t},a_{t},\Lambda_{t}\right)^{-\frac{1}{\sigma}}$. 
\end{enumerate}
Implementing this in an infinite horizon model I iterate from the intial guess until the difference $\left|V_{a,t}\left(\beta,e_{t},a_{t},\Lambda_{t}\right)-V_{a,t+1}\left(\beta,e_{t},a_{t},\Lambda_{t}\right)\right|$ is less than $\epsilon=1\times10^{-8}$, afterwards which the problem is assumed to have converged.  \\

The earnings process $e_t$ is assumed to follow an AR(1) process:
\begin{align*}
\log e_{t}=\rho\log e_{t-1}+\sigma^{e}\varepsilon, \varepsilon\backsim\mathcal{N}\left(0,1\right) 
\end{align*}
This implies that $e_t$ has mean 1, and follows a log-normal distribution. The process is discretized as a Markov chain using Rouwenhorst's method (which is preferable to Tauchen's method for very persistent processes) with $n^e=11$ points. The persistence parameter $\rho$ is picked from \citet{floden2001idiosyncratic} who estimates the earnings process for Sweden resulting in $\rho = 0.81$. The standard error is set to $0.92$ following \citet{kaplan2018monetary} (note though that this is based on American data!). The discretization results in a vector of states $e$ of size $n^e$, a stochastic transition matrix $\mathcal{D}^{e}$ of size $n^{e}\times n^{e}$. I let $\mathcal{D}^{e}\left(e^{i},e^{j}\right)$ denote the probability of transitioning from state $e^i$ to $e^j$. Iterating on the transition matrix yields the ergodic distribution vector $d^{e}$ (of size $1\times n^{e}$), where each entry is the share of households in state a particular state. Since the process has mean 1 we have $d^{e}\cdot e'=1$. \\

The distribution of discount factors follows a log-normal distribution with mean $\overline{\beta}$ and standard error $\sigma^{\beta}$. The process is discretized with $n^\beta=5$ points and state vector $\beta$. The parameters of the distribution are estimated using the method of simulated moments. The targeted moments are the wealth to after-tax labor income ratio, and the wealth shares of the bottom 50\%, middle 40\% and the top 10 \%. \\
 
To obtain the distribution of households over the two exogounes states $e,\beta$ and the endogenous state 
The distribution of households over the endogenous state $a$ is obtained as follows. Assume first the initial distribution $\mathcal{D}_{0}\left(\beta^{k},e^{i},a^{j}\right)=\frac{d^{e}\left(\beta^{k}\right)d^{e}\left(e^{i}\right)}{n^{a}}$ (probability of being in state $\beta^{k}$ and $e^{i}$ is $d^{e}\left(\beta^{k}\right)d^{e}\left(e^{i}\right)$, assume uniform distribution across assets). Iterate using the law of motion:
\begin{gather*}
\mathcal{D}_{t+1}\left(\beta^{k},e^{i},a^{j}\right)=\sum_{i'=1}^{n^{e}}\mathcal{D}^{e}\left(e^{i},e^{i'}\right)\sum_{j'=1}^{n^{a}}\mathcal{D}_{0}\left(\beta^{k},e^{i'},a^{j'}\right)\omega\left(a^{*}\left(e^{i'},a^{j'}\right),a^{\max\left\{ j-1,1\right\} },a^{j},a^{\min\left\{ j+1,n^{a}\right\} }\right),    
\end{gather*}
where:
\begin{gather*}
\omega\left(a^{*},\underline{a},\tilde{a},\overline{a}\right)=1\left\{ a^{*}\in\left[\underline{a},\overline{a}\right]\right\} \begin{cases}
\begin{array}{c}
\frac{\overline{a}-a^{*}}{\overline{a}-\tilde{a}},\\
\frac{a^{*}-\underline{a}}{\overline{a}-\underline{a}},
\end{array} & \begin{array}{c}
a\geq\tilde{a}\\
a<\tilde{a}
\end{array}\end{cases}
\end{gather*}
In words $\omega$ is linear interpolation of the endogenous state between the associated grid point and its two closest neighbours. Forward iteration of the law of motion until convergence gives the ergodic, steady-state distribution $\mathcal{D}_{ss}$. 
 
 
\section{Calibration}


For the households problem 




\bgroup
\def\arraystretch{1.0}
\setlength{\tabcolsep}{1.0em} 
\begin{table}[H]
\tiny
%\captionsetup{font=footnotesize}
\caption{Calibration} 
\label{GBB_components}
\centering
\footnotesize
\begin{threeparttable}
\scriptsize 
\makebox[\textwidth]{ 
\begin{tabular}{lcccc}
\toprule
\midrule
%\multicolumn{2}{c}{{Revenues}} & \multicolumn{2}{c}{Expenses} \\ 
 {Parameter}   & Desc.  & Value & Target  & Source  \\
 \cmidrule(lr){1-5}  
 
\multicolumn{5}{c}{\textit{Households}} \\
$\sigma$ & Intertemp. Elasticity of Substitution & 0.5 & - & Common Value \\ 
$\varphi$ & Disutility of labor & 0.01 & Job Finding rate = 30\% & \citet{hobijn2009job} \\ 
$\nu$ & Frisch Elasticity of Labor Supply & 2 & - &  \\ 

$\mu^\beta$ & Mean of Beta dist. & 0.93 & Bond-Labor Income ratio = 6.2 & Nationalbanken \\ 
$\sigma^\beta$ & SE of Beta dist. & 0.04 & Wealth Moments & - \\ 
$\rho^\e$ & Persistence of earnings & 0.81 & - & \citet{floden2001idiosyncratic} \\ 
$\sigma^\e$ & SE of earnings & 0.92 & - & - \\ 

\multicolumn{5}{c}{} \\ 
\multicolumn{5}{c}{\textit{Firms}} \\
$\delta^K$ & Capital deprecation &   &   &  \\ 
$I$ & SS Investment & 0.3  & 30\% of GDP  & Statistics Denmark  \\ 
\multicolumn{5}{c}{} \\ 
\multicolumn{5}{c}{\textit{Labor Market}} \\
$\kappa^V$ & Vacancy posting cost & - & SS unemployment rate = 5\% & Ministry of Finance \\ 
$\\xi$ & Matching Parameter & 0.5 & - & See\tnote{1} \\ 
$\mu^w$  & Union Market Power & 1.92 &  Replacement rate of 51\%  & IMF\tnote{2}  \\
$\delta^N$ & Job Destruction Rate & 0.06  & -  & \citet{hobijn2009job} \\ 

\multicolumn{5}{c}{} \\
\multicolumn{5}{c}{\textit{Government policy}} \\
$\tau^VAT$ & VAT & 0.25 &  -  &  -   \\
$\tau^c$ & Corporate Tax rate & 0.22 &  -  &  -   \\
$G$  & Public Consumption & 0.24 &  24\% of GDP  & Statistics Denmark   \\
%$b$  & Unemployment Benefits & 0.36 &  Replacement rate of 51\%  & IMF\tnote{2}  \\

\midrule[\heavyrulewidth]
\end{tabular}
}
\end{threeparttable}
\begin{tablenotes}[para]\footnotesize 
\item[1] The values for σ used in the literature are: 0.24 in Hall (2005a), 0.4 in Blanchard and Diamond (1989), Andolfatto
(1994) and Merz (1995), 0.45 in Mortensen and Nagypal (2006), 0.5 in Hagedorn and Manovskii (2006), 0.5 in Farmer
(2004), 0.72 in Shimer (2005a). See also a brief discussion in Mortensen and Nagypal (2006), p. 10, comparing their
value of 0.45 to Shimer’s one \\
\item[2] Labor Market Regulations in Low-, Middle- and High-Income Countries : A New Panel Database
\end{tablenotes}
\end{table}
