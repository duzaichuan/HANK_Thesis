


\section{Considerations}
\label{chap:Considerations}


\subsection{Government Budget Constraint}

Let $\tau$ denote a vector of taxes and $X$ a vector of associated tax bases. Let $G$ be public consumption, $b U$ unemployment benefits, $B$ nominal government bonds, and $r$ the rate of return. Finally, $p$ is the general price level.  The government's budget constraint is given by:
\begin{gather*}
\tau X_{t}+\frac{B_{t}}{p_{t}}=G_{t}+bU_{t}+T_{t}+\frac{B_{t-1}}{p_{t}}\left(1+r_{t}\right).    
\end{gather*}
The left hand side is revenue: It is composed of raised taxes, and the revenue that the sell of bonds give. The right hand is the expenditures: The government provides public consumption, unemployment benefits, transfers to households, and pays back bonds with interest. 

Let bars denote steady-state value. The governments budget is balanced in the long in the sense that the supply of real bonds and transfers converge to their steady state values by the rules: 
\begin{gather*}
\ln\left(g_{t}\right)=\ln\left(\bar{g}\right)-\gamma^{G}\ln\left(\frac{B_{t}/p_{t}}{\bar{B}}\right) \\
\ln T_{t}=\ln\left(\bar{T}\right)+\gamma^{T}\ln\left(\frac{B_{t}/p_{t}}{\bar{B}}\right)
\end{gather*}
The budget constraint reveals an obvious choice which the government is faced with continuously: To fund consumption and benefit programs using either non-distortionary taxes (transfers) or borrowing. 
Under certain assumptions (infinite horizon households, complete markets etc) this issue is non-existent since the two sources of funding has equivalent effects. This is the Ricardian Equivalence theorem (RET) (\citet{barro1974government}). 


\citet{elmendorf1999government}

Reasons for RET to not hold? 
Credit/borrowing constraints.
Income uncertainty/precautionary savings motive
%begin with the axioms that taxes are levied as
%a function of income and that furore income is uncertain. Therefore, when the
%overnment cuts taxes today, issues government debt, and raises income taxes in
%the future to pay off the debt, consumers' expected lifetime income is unchanged,
%but the uncertainty they face is reduced. If consumers have a precautionary saving
%motive, this reduction in uncertainty stimulates current consumption. Put differently,
%consumers discount risky uncertain income and uncertain future taxes at a higher
%rate than the interest rate on government bonds; a postponement of the tax burden,
%therefore, encourages current spending.
%incomplete markets? 