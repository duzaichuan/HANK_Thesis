





\section{Thesis Overview - Thoughts}
%\label{chap:res}


Empirical section: 
SVAR? 
\citet{blanchard2002empirical}, \citet{dolado1997causes}

Model section: 





Interesting thoughts and points in \citet{andersen2016automatic}. For example, interesting discussion of stabilizing effects of taxes versus the welfare losses they imply due to distortions (i.e. distort marginal incentives). 

Analysis: 
\begin{itemize}  
           \item Compare impulses responses of (for instance) productivity shock under different tax/transfer systems. 
           \item Compare volatility (variance) of aggregate variables under different tax/transfer systems. 
           \item Fiscal policy: Impulse responses of discretionary policy or compare impulses responses of productivity shock under different fiscal policy rules. 
           \item For fiscal policy, could follow \citet{eggertsson2012debt} (constrained consumers, fiscal policy that focuses on reliving debt written consumers, transfers of debt).  
\end{itemize}



\textbf{Zero Lower Bound}
A number of authors point out interest in automatic stabilizers in low interest rate environment. Interesting for a number of reasons. In a ZLB economy monetary policy is constrained, and cannot stimulate the economy any further (at least in traditional macro models) by lowering interest rates. Thus, other stabilization policies are needed -> discretionary fiscal policy and automatic stabilizers. Note also that these are more "effective" when at the ZLB since normally fiscal policy would raise inflation, which in turn would increase interest rates through the taylor rule, thus reducing demand. 
Another interesting consideration is that it is cheap for the government to finance spending by debt in a low rate environment. Could provide valuable discussion of financing: Taxes versus debt? \citet{blanchard2019public}, \citet{blanchard2020automatic}.



