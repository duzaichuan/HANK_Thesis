


\pagebreak 
\section{Additional Labor Market Frictions}

\citet{Jeppe2020}



\subsection{Model}
\subsubsection{Matching Process}
In the expanded model vacancies are modelled as a stock, and thus obeys the law of motion:
\begin{gather*}
V_{t}=\underline{V}_{t}+\iota_{t}, 
\end{gather*}
where $\underline{V}_{t}$ is the start-of-period stock of vacancies and $\iota_{t}$ is the number of new vacancies posted. $V_t$ is the relevant vacancy measure for matching, which still occurs through the Cobb-Douglas function $M_{t} = \mathcal{M} S_{t}^{\xi}V_{t}^{1-\xi}$. As before, let $m_t$ denote the matching probability and $q_t$ the job finding rate. The post-matching levels of unemployment and vacancies are:
\begin{gather*}
U_{t}=\left(1-q_{t}\right)S_{t}, \\
\tilde{V}_{t}=\left(1-m_{t}\right)V_{t}
\end{gather*}
Finally, an endogenous share of jobs $\delta^L_t$ are lost at the end of the period, and similarly vacancies are destroyed at an exogenous rate $\delta^V$. The number of searchers and vacancies at the start of the next period is then given by:
\begin{gather*}
S_{t+1}=U_{t}+\delta_{t}^{L}N_{t}, \\
\underline{V}_{t+1}=\left(1-\delta^{V}\right)\tilde{V}_{t}
\end{gather*}
 

\subsubsection{Matching Value}
Let $\mathcal{V}_{t}^{M}$ denote the time $t$ value of a match. At the end of the period each firm must pay a continuation cost $\chi_{t}$ to keep its match, or else it is destroyed. The continuation cost is stochastic and has cumulative distribution function $G$. Firms pay the contiuation cost if the expected value of keeping the match out weights the cost, $\frac{1}{1+r_{t+1}}\mathbb{E}_{t}\mathcal{V}_{t+1}^{M}>\chi_{t}$. Optimally entails that the value of a match is given by the bellman equation:
\begin{align*}
\mathcal{V}_{t}^{M}&=\frac{\partial div_{t}}{\partial N_{t}}+\int^{\chi_{c,t}}\left[\frac{1}{1+r_{t+1}}\mathbb{E}_{t}\mathcal{V}_{t+1}^{M}-\chi_{t}\right]dG\left(\chi_{t}\right) \\
&=mc_{t}\frac{\partial Y_{t}}{\partial N_{t}}-w_{t}+\int^{\chi_{c,t}}\left[\frac{1}{1+r_{t+1}}\mathbb{E}_{t}\mathcal{V}_{t+1}^{M}-\chi_{t}\right]dG\left(\chi_{t}\right),
\end{align*}
where $\chi_{c,t}=\frac{1}{1+r_{t+1}}\mathbb{E}_{t}\mathcal{V}_{t+1}^{M}$ is the cutoff above which firms do not pay the cost, and matches are destroyed. Accordingly, $G\left(\chi_{c,t}\right)$ is the number of matches kept for next period, such that $1-G\left(\chi_{c,t}\right)$ equals the endogenous job destruction rate, $\delta_{t}^{L}=1-G\left(\chi_{c,t}\right)$. Letting $\mu_{t}=\int^{\chi_{c,t}}\chi_{t}dG\left(\chi_{t}\right)$ denote the average continuation cost the Bellman equation can be restated as: 
\begin{gather*}
\mathcal{V}_{t}^{M}=mc_{t}\frac{\partial Y_{t}}{\partial N_{t}}-w_{t}+\frac{1-\delta_{t}^{L}}{1+r_{t+1}}\mathbb{E}_{t}\mathcal{V}_{t+1}^{M}-\mu_{t}
\end{gather*}


\subsubsection{Vacancy Creation}
The following Bellman equation describes the firm value of a vacancy post:
\begin{gather*}
\mathcal{V}_{t}^{V}=\mathcal{V}_{t}^{M}m_{t}-\kappa^{V}+\frac{\left(1-m_{t}\right)\left(1-\delta^{V}\right)}{1+r_{t+1}}\mathbb{E}_{t}\mathcal{V}_{t+1}^{V}
\end{gather*}
The model also contains entry costs for labor market firms. Each period a constant number of candidate firms $F$ draw a stochastic entry cost $c$ distributed in according to $H$. Firms pay the entry cost and pay vacancies if and only if $\mathcal{V}_{t}^{V}\geq c$, such that the number of new vacancies posted is:
\begin{gather*}
\iota_{t}=H\left(\mathcal{V}_{t}^{V}\right).
\end{gather*}