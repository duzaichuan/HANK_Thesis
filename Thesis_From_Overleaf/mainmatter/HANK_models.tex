




\chapter{Thoughts on Hank Models}
\section{Hank Models}
The New Keynesian (henceforth NK) school of though refers to the synthesis between the neo-classical school and the traditional Keynesian school of thought. In particular, the term refers encapsulates a range of economists and associated models that were developed in the 1980's as response to the neo-classical models developed in the 1960's and 1970's. The models were also aimed to be micro-founded to be robust against the Lucas critique, something that the traditional Keynesian models did not live up to. \\
Simple NK models typically resembles real business cycle models at the core, but are augmented with frictions such as price/wage rigidities and imperfect competition, which allows for short term responses to fiscal and monetary policy. In short, prices and wages are not flexible in the run. For instance, the standard 3-equation New Keynesian model includes monopolistic competition in the final goods sector and sticky prices through a Calvo mechanism. 
The model contains a Philips-curve (the New Keynesian Philips-curve), a dynamic IS equation, and a monetary policy rule:  
\begin{align*}
\pi_{t}&=\beta\mathbb{E}_{t}\pi_{t+1}+\kappa\hat{y}_{t} \\
\hat{y}_{t}&=\mathbb{E}_{t}\hat{y}_{t+1}-\frac{1}{\sigma}\left(\hat{i}_{t}-\mathbb{E}_{t}\pi_{t+1}\right)+u_{t} \\
\hat{i}_{t}&=f\left(\hat{y}_{t},\pi_{t}\right)+v_{t}
\end{align*}
This is an example of Representative agent NK model (RANK). A general issue with this model is that marginal propensity to consume (MPC) out of income is unreasonably low. Also, monetary policy works only through intertemporal substitution, which is also unreasonable. \\
The intuition regarding both (but primarily the low MPCs) is straightforward: The representative consumer models the \textit{average} consumer, who is usually reasonably well off in terms of income and wealth. This in turn implies a low MPC due to consumption smoothing. 
A straightforward, oftenly used extension is to add hand-to-mouth (HTM) consumers to the model thus giving a two agent NK (TANK) model. Hand to mouth consumers differ from rational agents in the sense that they simply consume their entire income each period. They thus add a much needed high MPC (of 1) to the model. 

The next natural line of models are the heterogeneous agent NK (HANK) models, who add a continuum of heterogeneous agents.  



\section{Hank and integrated SAM models}
%\subsection{Policy}
State of the art macroeconomic models (almost) all include a search and matching labor market ala Pissarides (\cite{pissarides2000equilibrium}, \cite{mortensen1994job}). This type of labor market model has also been integrated in a number of HANK model, see Challe (2019), Challe and Ragot (2016), den Haan et al (2018), McKay and Reis (2016b). From a Keynesian perspective, SAM labor markets contribute through searching and matching frictions which gives rise to frictional unemployment. 

Consider the case where one wishes to analyze fiscal policy and automatic stabilizers. What are the implications (i.e. gains and losses) of using a SAM labor market?   












\pagebreak
\subsection{Email} 

Hej Jeppe  

Jeg skal skrive speciale til efteråret, og leder i den forbindelse efter en vejleder. 
Jeg vil umiddelbart gerne skrive om finanspolitik/automatiske stabilisatorer i heterogene agent NK (HANK) modeller, inspireret af den nyere litteratur på området - jeg smider nogle referencer i bunden af mailen på de artikler, jeg har kigget mest på. 

Umiddelbart er der lidt numerisk/kode arbejde i projektet ifm. at løse et dynamisk programmeringsproblem og den generelle løsning af modellen, men det burde kunne lade sig gøre. Auclert et al. (2019) artiklen beskriver en rimelig effektiv metode til at løse den type af modeller. Koderne bag artiklen er offentligt tilgængelige og er til at arbejde videre på. Man kunne sandsynligvis skrive et helt speciale om efficient løsning af HANK modeller, men det er mit mål, at projektet skal være primært makro-orienret og mindre numerisk orienteret. 

Hvis du vurderer, at emnet falder inden for dit kompetenceområde (og du syntes det lyder interessant), kunne vi eventuelt snakke nærmere om det, når Corona-kaosset har lagt sig lidt (og universitet er åbent igen). 


Mvh. 
Nicolai Waldstrøm 


Referencer: \\
A. Auclert, B. Bard ́oczy, M. Rognlie, and L. Straub, “Using the sequence-space jacobian to solve and estimate heterogeneous-agent models,” National Bureau of Economic Research, Tech. Rep., 2019

A. Auclert,  M. Rognlie,  and L. Straub,  “The intertemporal keynesian cross,”National Bureau of Economic Research, Tech. Rep., 2018.

A.  McKay  and  R.  Reis,  “The  role  of  automatic  stabilizers  in  the  us  business cycle,”Econometrica, vol. 84, no. 1, pp. 141–194, 2016

A.  McKay  and  R.  Reis, “Optimal automatic stabilizers,” National Bureau of Economic Research,Tech. Rep., 2016.







% Title Policy in HANK \& SAM MODELS 