\pdfbookmark[0]{Abstract}{Abstract}
\chapter*{Abstract}
\label{chap:abstract}
Optic flow, the measure of the speed at which visual input moves past the eye, is a general feature of biological vision. As the performance of the visual system is thought to affect behaviour, we set out to investigate whether self-induced optic flow affects walking behaviour of the wood ant (\textit{Formica rufa}). Previous investigations of the use of optic flow in walking insects have been performed using freely moving animals, which does not permit full control of the visual experience. To overcome this we developed a novel virtual reality setup, which we validated by testing whether wood ants interact with a black beacon in virtual reality. We found that wood ants preferentially faced the beacon and exhibited aggression towards it. We take this to provide convincing evidence that wood ants interact with and utilise visual information within a virtual reality setup. To study if and how wood ants use self-generated optic flow to control walking behaviour they were placed in a completely white environment with a black pattern on the floor. We show that wood ants respond to decreased gain (x0.3) of self-induced translational optic flow by increasing their walking speed by 29-98\% and increasing their translational walking duration by 34-118\%, whereas neither normal or increased (3x) gain has any effects. We further show that the increased walking duration results both from increasing duration of walking bouts and decreasing duration of pauses. This is the first evidence that wood ants use self-induced optic flow, and the first direct evidence that the sensory experience affects locomotion duration and intermittency in any animal.